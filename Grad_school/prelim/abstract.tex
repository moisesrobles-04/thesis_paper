



%__________   ABSTRACT ENGLISH ________________________________
\vspace*{0.5in}
\begin{center}
\section*{ABSTRACT}
\end{center}
\addcontentsline{toc}{section}{ABSTRACT} %para que aparezca en la tabla de contenido

\noindent
Combating disinformation in social media is a critical problem, notably when the disinformation targets healthcare. We explore how to fine-tune Large Language Models (LLM) to counteract health-related disinformation on social media. The fine-tuned base models for this project are T5, BERT, and LlaMa-2. We divide the fine-tuning into two sections: 1) classifying if the text is health-related and 2) verifying if the text contains disinformation. To rebut disinformation we use Retrieval Augmented Generation (RAG) to query trusted medical sources. Our experiment shows that the models can classify health-related with 94\% precision, 95\% recall, and 90\% F1. We also show that we classify disinformation texts with 99\% precision, 95\% recall, and 97\% F1. We present a project that can help health experts combat and rebut disinformation on different social media platforms.

%____________________________________________________________





\newpage




%__________   ABSTRACT ESPANOL  ______________________________

\vspace*{0.5in}
\begin{center}
\section*{RESUMEN}
\end{center}
\addcontentsline{toc}{section}{RESUMEN} %para que aparezca en la tabla de contenido

\noindent
El Resumen debe ser una traduccion del Abstract. No deben diferir en contenido. 
%____________________________________________________________