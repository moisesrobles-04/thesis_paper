
\section{Introduction}
\noindent Nowadays, technology has advanced to the point that anyone can find any information in just a few seconds. Social media has been an essential
element in the search for information. The issue with this is that anyone can find, search, share, and even write anything, accurate or not. However, this dilemma
has caused problems in this modern era. If anyone can share anything, how can you be sure what is true? Users are susceptible to disinformation or misinformation.
In this context, {\em misinformation} refers to messages with false information dispersed because the author misunderstood facts. In contrast, {\em disinformation} refers to
messages with false information that are intentionally dispersed. The author of these messages has the intention of forming opinions based on false data. In either
case, false information spreads to readers as facts. Most social media platforms recommend that users read from experts or official news outlets. Nevertheless,
the overwhelming amount of data makes it complicated to keep up with everything.

Currently, social media such as X, 
have ``Community Notes" which clarify tweets that are misleading or misinforming. However, this system
depends totally on human interaction and is a slow and intricate process, unable to keep up with the disinformation effort.
The issue of detecting and preventing the spread of misinformation has not been an easy task, especially in the health field.  In recent times, it has
been challenging for health officials to manage pandemics. Often, these officials need to make educational
campaigns for the population. However, social media misinformation can reduce the effectiveness of these campaigns \cite{article}. 
Some problems these experts have faced in the past years were misinformation
about vaccines, and social distancing measures.

In this paper, we present enhancements to  the Twitter Health Surveillance (THS) system to detect and rebut misinformation. 
THS is a prototype we are building at the University
of Puerto Rico, Mayag\"uez (UPRM), designed to detect tweets related to health conditions \cite{8622504}. THS is an integrated platform to help health officials collect tweets, determine if they are related to a medical condition, extract
metadata from them, and create a warehouse that can be used to analyze the data further.  THS has Artificial Intelligence (AI) components based on Long-Short Term Memory (LSTM)
and Gated Recurrent Unit (GRU), used to classify tweets as being medically related.

In this work, we use prior data from the THS project as our primary source for the health classification dataset. Likewise, the misinformation dataset contains social media posts, articles,
websites, and others \cite{stephencrone2022,coviddata,covidunesco}. 
However, we explore  Large Language Models (LLM)  to detect health misinformation, provide context for its classification decisions, and provide a rebuttal to the tweet.  A key feature is the use of 
curated medical datasets and Retrieval Augmented Generation (RAG) to help the LLM generate a rebuttal to the misinformation.
We built our new prototype components using PyTorch, Chroma, Ollama, and other open-source tools. For the task of determining if a text is health-related, our system achieved a 90\% F1 score. In addition, it achieved a 97\% F1 for misinformation detection. Additionally, our initial results show that querying  official health sources with RAG helps the LLM rebut appropriately  with an F1 BertScore of 82\%. This illustrates that it is possible to use LLM to fight health-related misinformation. 

\subsection{Contributions}
This paper provides the following original contributions:
\begin{itemize}
	\item{\textbf{Using LLMs for Health Misinformation Detection:}} 
	 We present LLM as a solution to classify and rebut health misinformation texts on social media and use research papers extracted from the NIH PubMed site as context for the LLM.
	\item{\textbf{Present a novel solution to misinformation rebuttal:}} To rebut misinformation, it is necessary to have an understanding of what needs to be fact-checked. 
	We extracted biomedical research papers and added them to a vector database. This setup enable us to use RAG to rebut health misinformation with peer-review documents.
	\item{\textbf{Pipeline Interface:}} We developed a frontend application that showcase the full pipeline, allowing users to interact with THS.
\end{itemize}


\subsection{Paper Organization}
Section II contains background on LLM  architectures, vector databases, and THS. In section III, we present the system architecture for the data extraction, classification process, and user interface. Later, in section IV we show the system performance. Ending with section V, we have our conclusions.
